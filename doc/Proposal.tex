\title{Exploring weaker atomic instructions for latch-free data
structures}
\date{\vspace{-5ex}}

\documentclass{sig-alternate}

\begin{document}
\numberofauthors{1} 

\author {
\alignauthor
Shengliang Xu\\
\email{slxu@cs.washington.edu}
}
\maketitle

\section{Introduction}

Since the rise of multi-core architectures in the recent years,
multi-threaded concurrent programming quickly become one of the
critical programming techniques for further improving program
performances. 

The biggest problem in concurrent programming is synchronizing access
to shared memories among different threads. There are generally two
ways of synchronization. 

\begin{itemize} 

  \item The first one is blocking through locks. Before accessing an
    area of shared memories, a thread must first aquire a lock. If the
    lock is currently held by another thread, the aquiring thread will
    yield the CPU. OS is responsible for awaking threads who are
    chosen for execution after the lock is free.  

  \item The second one is non-blocking synchronization through special
    atomic instructions. The de facto atomic instruction is the
    Compare-and-Swap (CAS). Most commonly, CAS is used to implement a
    spin lock. When a thread wants to access a shared memory block, it
    first gets a spin lock.  Other threads wait the lock to be free by
    repeatedly testing the spin lock.

\end{itemize}

Atomic instruction based non-blocking synchronization has the
advantage of low system overhead comparing with blocking
synchronizations. Recently, a lot of research efforts have been put
into developing latch-free data structures. And also, many higher
level programming languages add direct support to them through
libraries , such as Java ( in the java.util.concurrent.atomic package). 


\begin{table}
  \label{table:atomic_instructions}
  \caption{Atomic instructions provided by various platforms}
  \begin{tabular}{ l || p{1.5cm} | p{1.5cm} | p{1.2cm} | p{1.2cm} }
    & Compare-and-Swap & Swap & Test-and-Set & Fetch-and-Add \\ \hline
    ARM & LL/SC & deprecated & no & no \\
    POWER & LL/SC & no & no & no \\
    SPARC & yes & deprecated & yes & no \\
    x86 & yes & yes& yes & yes \\ \hline
  \end{tabular}
\end{table}


There are several different atomic instructions available in current
CPU architectures, with different semantics. Table
\ref{table:atomic_instructions} lists the most popular, if not all of,
atomic instructions. Among them, CAS has the strongest functionality
because we are able to use CAS to replace all others. Most of research
and engineering on latch free data structure and problem solving are
based on CAS. For example, OpenJDK 9 uses only CAS to implement all
its atomic variable functionalities.

How CAS works is that (this is how X86 implemented it) given a memory
word $m_1$, compare the value of it with another value $c$, if the two
matches, replace the value of $m_1$ to a new value $m_2$.  The problem
of CAS is that it can fail. The typical usage of CAS is using a while
loop such as the following simple code block:

\begin{quote}
  while (CAS fail) \{
    ...
  \} 
\end{quote}

It basically means keeping doing CAS until it succeeds. This can
result in poor performances if the number of threads concurrently
doing CAS is big.


On the other hand, the weaker atomic instructions succeed every time.
This difference may result in highly efficient algorithms for some
concurrent problems. For example, for an atomic integer ID generator,
Fetch-and-Add should be able to outperform Compare-and-Swap a lot
in high contention situations. Recently, Morrison et. al. designed a
new FIFO queue using Fetch-and-Add \cite{Morrison:2013} which
dramatically outperforms CAS based and blocking locking based FIFO
queues in high contention situations.


\section{Milestones}

Week 1~2: Add weak atomic instruction support to OpenJDK 9. And test
the performance 

Week 3~4: Find a data structure that is suitable for implementation
using weaker atomic instructions. 

Week 5~: Implement the chosen data structure.

\bibliographystyle{abbrv}
\bibliography{CSE548}


\end{document}
