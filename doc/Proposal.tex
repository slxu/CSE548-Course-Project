\title{Exploring weaker atomic instructions for latch-free data
structures}
\date{\vspace{-5ex}}

\documentclass{sig-alternate}

\begin{document}
\numberofauthors{1} %  in this sample file, there are a *total*
% of EIGHT authors. SIX appear on the 'first-page' (for formatting
% reasons) and the remaining two appear in the \additionalauthors section.
%
\author{
% You can go ahead and credit any number of authors here,
% e.g. one 'row of three' or two rows (consisting of one row of three
% and a second row of one, two or three).
%
% The command \alignauthor (no curly braces needed) should
% precede each author name, affiliation/snail-mail address and
% e-mail address. Additionally, tag each line of
% affiliation/address with \affaddr, and tag the
% e-mail address with \email.
%
% 1st. author
\alignauthor
Shengliang Xu\\
       \email{slxu@cs.washington.edu}
}
\maketitle

%\category{H.4}{Information Systems Applications}{Miscellaneous}
%\category{D.2.8}{Software Engineering}{Metrics}[complexity measures, performance measures]
%\terms{Theory}
%\keywords{ACM proceedings, \LaTeX, text tagging}

\section{Introduction}

Since the rise of multi-core architectures, multi-threaded concurrent
programming quickly become the technique for further improving program
performances. 

The biggest problem in concurrent programming is to synchronize access
of shared memories. There are generally two ways of synchronization
mechnisms. The first one is blocking through locks. Before accessing
an area of shared memories, a thread must first aquire a lock. If the
lock is currently held by another thread, the aquiring thread will
yield the CPU and wait for the lock.  The second one is non-blocking
synchronization through special atomic instructions. The de facto
atomic instruction is the Compare-and-Swap (CAS). Most commonly, when
a thread wants to access a shared memory block, it first uses an
atomic instruction to atomically change a small piece of memory to an
expected state. Other threads see the change and so won't break the
thread's operations.

Atomic instruction based non-blocking synchronization has the
advantage of low system overhead comparing with blocking
synchronizations. Recently, a lot of research efforts have been put
into developing latch-free data structures. However, most of them are
based on the strongest atomic instruction, i.e. CAS. For example,
OpenJDK 9 uses only CAS to implement all its atomic functionalities
(The java.util.concurrent.atomic).

Except for CAS, there are several weaker atomic instructions, as
listed in Table \ref{table:atomic_instructions}. CAS is the strongest
atomic instruction. It is able to implement all other instructions.
What it does is (this is how X86 implemented it) given a memory word
$m_1$, compare the value of it with another value $c$, if the two
matches, replace the value of $m_1$ to a new value $m_2$. 

The problem of CAS is that it can fail. The typical usage of CAS is
using a while loop such as the following simple code block:

\begin{quote}
  while (CAS fail) \{
    ...
  \} 
\end{quote}

It basically means keeping doing CAS until it succeeds. This can
result in poor performances if the number of threads concurrently
doing CAS is big.

On the other hand, the weaker atomic instructions succeed every time.
This difference may result in highly efficient algorithms for some
concurrent problems. For example, for an atomic integer ID generator,
Fetch-and-Add should be able to outperform Compare-and-Swap a lot
in high contention situations. Recently, Morrison et. al. designed a
new FIFO queue using Fetch-and-Add \cite{Morrison:2013} which
dramatically outperforms CAS based and blocking locking based FIFO
queues in high contention situations.

\begin{table}
  \label{table:atomic_instructions}
  \caption{Atomic instructions provided by various platforms}
  \begin{tabular}{ l || c  c  c  c }\hline
    & Compare-and-Swap & Swap & Test-and-Set & Fetch-and-Add \\ \hline
    ARM & LL/SC & deprecated & no & no \\
    POWER & LL/SC & no & no & no \\
    SPARC & yes & deprecated & yes & no \\
    x86 & yes & yes& yes & yes \\ \hline
    \end{tabular}
\end{table}


\section{Milestones}

Week 1~2: Add weak atomic instruction support to OpenJDK 9. And test
the performance 

Week 3~4: Find a data structure that is suitable for implementation
using weaker atomic instructions. 

Week 5~: Implement the chosen data structure.

\bibliographystyle{abbrv}
\bibliography{CSE548}


\end{document}
